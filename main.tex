\documentclass[aspectratio=169]{beamer}
\usepackage{customAmurmaple} % Include the custom Amurmaple theme

% Packages
\usepackage{amsmath}
\usepackage{amssymb}
\usepackage{physics}
\usepackage{graphicx}
\usepackage{tikz}
\usepackage{bm}
\usepackage{hyperref}
\usepackage{animate}
\usepackage[font=tiny]{caption}
\usepackage[backend=biber, style=numeric]{biblatex}
\addbibresource{references.bib}

% Title Information
\title[Adiabatic Theorem]{Adiabatic Theorem and Quantum Annealing}
\subtitle{A Mathematically Rigorous Treatment}
\author{Niall, Fintan, Shane}
\date{\today}
\institute{Applied Quantum Mechanics}
\titlegraphic{\includegraphics[width=2cm]{figures/uni.png}}

% Custom commands
\newcommand{\Hhat}{\hat{H}}
\newcommand{\psiket}{\ket{\psi}}
\newcommand{\dt}{\frac{d}{dt}}

\begin{document}

% Title Slide
\begin{frame}
    \titlepage
\end{frame}

% Table of Contents
\begin{frame}{Outline}
    \tableofcontents
\end{frame}

% Introduction
\section{Introduction}

\begin{frame}{Historical Context}
    \begin{columns}
    \hspace{0.5cm}
    \column{0.6\textwidth}
    \begin{block}{The beginning of quantum theory}
        \centering
        
        Einstein 1911 gave Solvay Conference on quantum hypothesis $E=nhv$ for atomic oscillators. 
    \end{block}  
        \vspace{0.5cm}

        \centering
        \begin{figure}
            \centering
            \includegraphics[width=0.8\textwidth]{figures/ECorrection.png}
            \caption{\tiny{Einstein Lorentz Debate Catoon \textit{Vecteezy}}} 
        \end{figure}

    \column{0.4\textwidth}
        \begin{figure}
            \centering
            \includegraphics[width=\textwidth]{figures/Solvay_conference.png}
            \caption{\tiny{1911 Solvay Conference \textit{Wikimedia Commons}}}
        \end{figure}

    \end{columns}
\end{frame}


\begin{frame}{Historical Context}
    \begin{columns}
    \hspace{0.5cm}
    \column{0.7\textwidth}

    \begin{block}{Classical Analogy}
        A pendulum whose length changes slowly will maintain its oscillation mode, but if changed rapidly, it will exhibit chaotic behavior.
    \end{block}

    \vspace{0.5cm}

    Also called adiabtic invariant, written by Ehrenfest in 1916 \cite{Ehrenfest1916}, and extended to quantum mechanics by Born and Fock in 1928 \cite{Born1928}.

    \column{0.3\textwidth}
    \begin{figure}
        \centering
        \includegraphics[width=0.8\textwidth]{figures/Adiabatic-pendulum.png}
        \caption{\tiny Pendulum with Varying Length \textit{Wikimedia Commons}}
    \end{figure}
    \end{columns}
\end{frame}

\begin{frame}{What is the Adiabatic Theorem?}
    \begin{columns}
    \hspace{0.5cm}
    \column{0.6\textwidth}
    \begin{block}{Physical Statement}
        If a quantum system starts in an eigenstate of a time-dependent Hamiltonian, and if the Hamiltonian changes \textbf{sufficiently slowly}, the system will remain in the corresponding instantaneous eigenstate throughout the evolution.
    \end{block}

    \vspace{0.15cm}

    \begin{itemize}
        \item Key word: \textbf{``sufficiently slowly''}
        \item The system acquires a \textbf{dynamical phase} and a \textbf{geometric phase} (Berry phase)
        \item Fundamental to quantum computing (adiabatic quantum computation)
        \item Applies to quantum chemistry, biology / condensed matter physics
    \end{itemize}
    
    \column{0.4\textwidth}
    \centering
    \begin{figure}
        \centering
        \animategraphics[autoplay, loop, width=\textwidth]{20}{figures/frames/slow_adiabatic_potential_animation_}{000}{099}
        \caption{\tiny Slow change (animated)}
    \end{figure}
    \begin{figure}
        \centering
        \animategraphics[autoplay, loop, width=\textwidth]{15}{figures/frames_adiabatic/adiabatic_potential_animation_}{000}{083}
        \caption{\tiny Abrupt change (animated)}
    \end{figure}
    \end{columns}
\end{frame}

% Mathematical Framework
\section{Mathematical Framework}

\begin{frame}{Time-Dependent Schrödinger Equation}
    Consider a time-dependent Hamiltonian $\Hhat(t)$:
    \begin{equation}
        i\hbar \frac{\partial}{\partial t}\ket{\psi(t)} = \Hhat(t)\ket{\psi(t)}
    \end{equation}

    \vspace{0.3cm}

    \begin{block}{Instantaneous Eigenstates}
        At each time $t$, we can solve the eigenvalue problem:
        \begin{equation}
            \Hhat(t)\ket{n(t)} = E_n(t)\ket{n(t)}
        \end{equation}
        where $\ket{n(t)}$ are the \textbf{instantaneous eigenstates} and $E_n(t)$ are the \textbf{instantaneous eigenvalues}.
    \end{block}

    \vspace{0.3cm}

    \textbf{Note:} These are \emph{not} solutions to the time-dependent Schrödinger equation, but rather eigenstates at fixed time $t$.
\end{frame}

%\begin{frame}{Skip?: Orthonormality and Completeness}
%    The instantaneous eigenstates satisfy:
%
%    \begin{block}{Orthonormality}
%        \begin{equation}
%            \braket{n(t)}{m(t)} = \delta_{nm}
%        \end{equation}
%    \end{block}
%
%    \begin{block}{Completeness}
%        \begin{equation}
%            \sum_n \ket{n(t)}\bra{n(t)} = \mathbb{I}
%        \end{equation}
%    \end{block}
%
%    \vspace{0.3cm}
%
%    Any state can be expanded in this instantaneous basis:
%    \begin{equation}
%        \ket{\psi(t)} = \sum_n c_n(t) e^{i\theta_n(t)} \ket{n(t)}
%    \end{equation}
%
%    where $c_n(t)$ are complex coefficients and $\theta_n(t)$ are phases to be determined.
%\end{frame}

% Derivation of the Adiabatic Theorem
\section{Derivation of the Adiabatic Theorem}

\begin{frame}{Substituting into Schrödinger Equation}
    Substitute the expansion $\ket{\psi(t)} = \sum_n c_n(t) e^{i\theta_n(t)} \ket{n(t)}$ into the Schrödinger equation:
    \begin{multline}
        i\hbar \frac{\partial}{\partial t}\ket{\psi(t)} = \Hhat(t)\ket{\psi(t)} \\
        i\hbar \sum_n \left[\dot{c}_n e^{i\theta_n}\ket{n} + c_n i\dot{\theta}_n e^{i\theta_n}\ket{n} + c_n e^{i\theta_n}\ket{\dot{n}}\right] = \sum_n c_n e^{i\theta_n} E_n(t)\ket{n}
    \end{multline}

    \vspace{0.3cm}

    Dividing by $i\hbar$:
    \begin{equation}
        \sum_n \left[\frac{\dot{c}_n}{i\hbar} e^{i\theta_n}\ket{n} - \frac{c_n\dot{\theta}_n}{\hbar} e^{i\theta_n}\ket{n} + \frac{c_n}{i\hbar} e^{i\theta_n}\ket{\dot{n}}\right] = \sum_n \frac{c_n E_n}{i\hbar} e^{i\theta_n}\ket{n}
    \end{equation}
\end{frame}

\begin{frame}{Projecting onto Eigenstates}
    Take the inner product with $\bra{m(t)}$:
    
    \begin{equation}
        \frac{\dot{c}_m}{i\hbar} e^{i\theta_m} - \frac{c_m\dot{\theta}_m}{\hbar} e^{i\theta_m} + \frac{c_m}{i\hbar} e^{i\theta_m}\braket{m}{\dot{m}} = \frac{c_m E_m}{i\hbar} e^{i\theta_m}
    \end{equation}

    \vspace{0.3cm}
    
    We used orthonormality: $\braket{m}{n} = \delta_{mn}$ so that $\bra{m}\Hhat\ket{n} = E_m\delta_{mn}$.
    
    \vspace{0.3cm}
    
    Multiplying by $i\hbar$ and rearranging:
    \begin{equation}
        \dot{c}_m e^{i\theta_m} = c_m e^{i\theta_m}\left[i\hbar\dot{\theta}_m - E_m + i\hbar\braket{m}{\dot{m}}\right] - \sum_{n\neq m} c_n e^{i\theta_n}\braket{m}{\dot{n}}
    \end{equation}
\end{frame}

\begin{frame}{Simplifying Terms}
    \begin{columns}
    \hspace{0.5cm}
    \column{0.6\textwidth}
    \begin{block}{Dynamical Phase}
        \begin{equation}
            \theta_m(t) = -\frac{1}{\hbar}\int_0^t E_m(t')dt' + \gamma_m(t)
        \end{equation}
    \end{block}
    
    where $\gamma_m(t)$ is the \textbf{geometric phase} (Berry phase):
    
    \begin{block}{Geometric Phase}
        \begin{equation}
            \gamma_m(t) = i\int_0^t \braket{m(t')}{\dot{m}(t')}dt'
        \end{equation}
    \end{block}
    
    \vspace{0.1cm}
    
    Then: $\dot{\theta}_m = -\frac{E_m}{\hbar} + i\braket{m}{\dot{m}}$
    
    \vspace{0.1cm}
    
    Substituting back: $i\hbar\dot{\theta}_m - E_m + i\hbar\braket{m}{\dot{m}} = 0$, so:
    \begin{equation}
        \dot{c}_m e^{i\theta_m} = - \sum_{n\neq m} c_n e^{i\theta_n}\braket{m}{\dot{n}}
    \end{equation}

    \column{0.4\textwidth}
        \begin{figure}
            \centering
            \includegraphics[width=\textwidth]{figures/BerryBoy.jpg}
            \caption{\tiny Michael Berry \textit{Wikimedia Commons}}
        \end{figure}
    \end{columns}
\end{frame}

\begin{frame}{What is $\braket{m}{\dot{n}}$}
    From the eigenvalue equation $\Hhat(t)\ket{n(t)} = E_n(t)\ket{n(t)}$, differentiate with respect to $t$:
    
    \begin{equation}
        \dot{\Hhat}\ket{n} + \Hhat\ket{\dot{n}} = \dot{E}_n\ket{n} + E_n\ket{\dot{n}}
    \end{equation}
    
    Take inner product with $\bra{m}$ (where $m \neq n$):
    
    \begin{equation}
        \bra{m}\dot{\Hhat}\ket{n} + \bra{m}\Hhat\ket{\dot{n}} = \dot{E}_n\braket{m}{n} + E_n\braket{m}{\dot{n}}
    \end{equation}
    
    Since $\bra{m}\Hhat = E_m\bra{m}$ and $\braket{m}{n} = 0$ for $m \neq n$:
    
    \begin{equation}
        \bra{m}\dot{\Hhat}\ket{n} + E_m\braket{m}{\dot{n}} = E_n\braket{m}{\dot{n}}
    \end{equation}

    \begin{block}{Matrix Element}
        \begin{equation}
            \braket{m}{\dot{n}} = \frac{\bra{m}\dot{\Hhat}\ket{n}}{E_n - E_m} \quad (m \neq n)
        \end{equation}
    \end{block}
\end{frame}

%\begin{frame}{The Differential Equation for Coefficients}
%    Substituting back:
%    
%    \begin{equation}
%        \dot{c}_m = - \sum_{n\neq m} c_n e^{i(\theta_n - \theta_m)} \frac{\bra{m}\dot{\Hhat}\ket{n}}{E_n - E_m}
%    \end{equation}
%    
%    The phase difference is:
%    \begin{equation}
%        \theta_n - \theta_m = -\frac{1}{\hbar}\int_0^t [E_n(t') - E_m(t')]dt' + [\gamma_n(t) - \gamma_m(t)]
%    \end{equation}
%    
%    \vspace{0.3cm}
%    
%    \begin{alertblock}{Key Observation}
%        The right-hand side contains:
%        \begin{itemize}
%            \item $\dot{\Hhat}$ - rate of change of Hamiltonian
%            \item $E_n - E_m$ - energy gap (in denominator)
%            \item Rapidly oscillating phase factor
%        \end{itemize}
%    \end{alertblock}
%\end{frame}

\begin{frame}{The Adiabatic Condition}
    For the system to remain in state $\ket{m}$, we need $\dot{c}_m \approx 0$ for $m$ equal to the initial state.
    
    \vspace{0.3cm}
    
    \begin{block}{Adiabatic Condition}
        The Hamiltonian changes slowly enough that:
        \begin{equation}
            \left|\frac{\bra{m}\dot{\Hhat}\ket{n}}{E_n - E_m}\right| \ll \frac{|E_n - E_m|}{\hbar}
        \end{equation}
        for all $n \neq m$.
    \end{block}
    
    \vspace{0.3cm}
    
    This can be rewritten as:
    \begin{equation}
        \left|\bra{m}\dot{\Hhat}\ket{n}\right| \ll \frac{(E_n - E_m)^2}{\hbar}
    \end{equation}
    
    \vspace{0.3cm}
    
    \textbf{Physical Interpretation:} The time scale of Hamiltonian variation must be much larger than $\hbar/\Delta E$, where $\Delta E$ is the relevant energy gap.
\end{frame}

% The Adiabatic Theorem
\section{The Adiabatic Theorem}

\begin{frame}{Formal Statement}
    \begin{theorem}[Quantum Adiabatic Theorem]
        Let $\Hhat(t)$ be a time-dependent Hamiltonian with instantaneous eigenstates $\ket{n(t)}$ and eigenvalues $E_n(t)$. Suppose:
        \begin{enumerate}
            \item The system starts in eigenstate $\ket{n(0)}$: $\ket{\psi(0)} = \ket{n(0)}$
            \item The eigenvalues are non-degenerate
            \item The adiabatic condition is satisfied for all $t \in [0, T]$
        \end{enumerate}
        Then the state at time $t$ is \cite{Kato1950, Griffiths2018}:
        \begin{equation}
            \ket{\psi(t)} = e^{i\theta_n(t)}\ket{n(t)} + \mathcal{O}(\epsilon)
        \end{equation}
        where $\epsilon$ characterizes the adiabatic parameter and
        \begin{equation}
            \theta_n(t) = -\frac{1}{\hbar}\int_0^t E_n(t')dt' + i\int_0^t\braket{n(t')}{\partial_{t'}n(t')}dt'
        \end{equation}
    \end{theorem}
\end{frame}

%\begin{frame}{Components of the Phase}
%    The total phase $\theta_n(t)$ has two parts:
%    
%    \begin{block}{1. Dynamical Phase}
%        \begin{equation}
%            \theta_n^{\text{dyn}}(t) = -\frac{1}{\hbar}\int_0^t E_n(t')dt'
%        \end{equation}
%        This is the familiar time evolution phase from energy.
%    \end{block}
%    
%    \begin{block}{2. Geometric Phase (Berry Phase)}
%        \begin{equation}
%            \gamma_n(t) = i\int_0^t\braket{n(t')}{\partial_{t'}n(t')}dt'
%        \end{equation}
%        This depends only on the \textbf{path} taken through parameter space, not on the rate of traversal.
%    \end{block}
%    
%    \vspace{0.3cm}
%    
%    For a cyclic evolution where $\ket{n(T)} = e^{i\alpha}\ket{n(0)}$:
%    \begin{equation}
%        \gamma_n = i\oint \braket{n}{\nabla_R n} \cdot dR
%    \end{equation}
%    where $R$ represents the parameters in $\Hhat(R)$.
%\end{frame}

% Examples and Applications
\section{Examples and Applications}

\begin{frame}{Example 1: Spin-1/2 in Rotating Magnetic Field}
    \begin{columns}
    \hspace{0.25cm}
    \column{0.725\textwidth}
    Consider a spin-1/2 particle in a magnetic field that rotates slowly:
    
    \begin{equation}
        \vec{B}(t) = B_0(\sin\theta\cos\omega t, \sin\theta\sin\omega t, \cos\theta)
    \end{equation}
    
    The Hamiltonian is:
    \begin{equation}
        \Hhat(t) = -\gamma \vec{B}(t) \cdot \vec{\sigma} = -\gamma B_0 \begin{pmatrix} \cos\theta & \sin\theta e^{-i\omega t} \\ \sin\theta e^{i\omega t} & -\cos\theta \end{pmatrix}
    \end{equation}
    
    Instantaneous eigenvalues: $E_{\pm} = \mp \gamma B_0$
    
    \vspace{0.3cm}
    
    If the spin starts aligned with $\vec{B}(0)$ and $\omega$ is small enough:
    \begin{itemize}
        \item The spin remains aligned with $\vec{B}(t)$ (adiabatic following)
        \item After one full rotation ($t = 2\pi/\omega$), acquires Berry phase: $\gamma = \pi(1 - \cos\theta)$
    \end{itemize}

    \column{0.3\textwidth}
    \centering
    \begin{figure}
        \centering
        \animategraphics[autoplay, loop, width=\textwidth]{15}{figures/frames_berry_phase/frame_}{0000}{0119}
        \caption{\tiny Berry phase evolution (animated)}
    \end{figure}
    \end{columns}

\end{frame}

\begin{frame}{Example 2: Two-Level System (Landau-Zener Model) \cite{Landau1932, Zener1932}}
    \begin{columns}
    \hspace{0.25cm}
    \column{0.5\textwidth}

    Consider the Landau-Zener Hamiltonian:
    \begin{equation}
        \Hhat(t) = -\Delta \sigma_x - \epsilon(t) \sigma_z
    \end{equation}
    
    where $\epsilon(t)$ is swept linearly from $\epsilon_i < 0$ to $\epsilon_f > 0$.
    
    \vspace{0.3cm}
    
    Energy eigenvalues:
    \begin{equation}
        E_{\pm}(t) = \pm\sqrt{\Delta^2 + \epsilon^2(t)}
    \end{equation}
    
    Energy gap at crossing ($\epsilon = 0$): $\Delta E_{\min} = 2\Delta$
    
    \column{0.5\textwidth}
    \centering
    \begin{figure}
        \centering
        \animategraphics[autoplay, loop, width=\textwidth]{15}{figures/frames_lz_no_bloch/frame_}{0000}{0119}
        \caption{\tiny Landau--Zener dynamics (animated)}
    \end{figure}
    \end{columns}
\end{frame}

%\begin{frame}{Landau-Zener Model}
%    \begin{columns}
%    \hspace{0.25cm}
%    \column{0.55\textwidth}
%    
%    \begin{itemize}
%        \item The energy levels exhibit an \textbf{avoided crossing} at $\epsilon(t) = 0$
%        \item Minimum gap: $\Delta E_{\min} = 2\Delta$ determines adiabatic time scale
%    \end{itemize}
%
%    \vspace{0.15cm}
%    
%    \begin{block}{Adiabatic Condition}
%        \begin{equation}
%            T \gg \frac{\hbar}{\Delta E_{\min}} = \frac{\hbar}{2\Delta} \quad \text{and} \quad \left|\dot{\epsilon}\right| \ll \frac{4\Delta^2}{\hbar}
%        \end{equation}
%    \end{block}
%    
%    This is the canonical model for avoided crossings and adiabatic transitions!
%    \column{0.45\textwidth}
%    \centering
%    \begin{figure}
%        \centering
%        \includegraphics[width=\textwidth]{figures/avoided_crossing.png}
%        \caption{\tiny Avoided crossing energy levels}
%    \end{figure}
%    \end{columns}
%\end{frame}
%\begin{frame}{Discussion: Avoided Crossing Physics}
%    \begin{block}{Key Questions}
%        \begin{enumerate}
%            \item What happens at the avoided crossing point?
%            \item Why is this the bottleneck for adiabatic evolution?
%            \item How does the gap size affect the required evolution time?
%        \end{enumerate}%
%    \end{block}
%    
%    \vspace{0.3cm}
%    
%    \textbf{Physical Insights:}
%    \begin{itemize}
%        \item At $\epsilon = 0$: The eigenstates are equal superpositions of diabatic states
%        \item The gap $2\Delta$ prevents level crossing (quantum mechanical effect)
%        \item Smaller gap $\Rightarrow$ easier to violate adiabaticity
%        \item Without the $-\Delta\sigma_x$ term, levels would cross diabatically
%    \end{itemize}
%    
%    \vspace{0.3cm}
%    
%    \begin{alertblock}{Critical Insight}
%        The adiabatic condition is hardest to satisfy near the \textbf{minimum gap}. This is where non-adiabatic transitions (Landau-Zener tunneling) are most likely to occur.
%    \end{alertblock}
%\end{frame}


% Section 6: Geometric Phase
\section{Geometric Phase (Berry Phase)}

\begin{frame}{Berry's Geometric Phase}
    For a cyclic adiabatic evolution, where the Hamiltonian returns to its initial form:
    
    \begin{equation}
        \Hhat(R(T)) = \Hhat(R(0))
    \end{equation}
    
    The total phase acquired is:
    \begin{equation}
        e^{i\phi} = e^{i\theta_{\text{dyn}}} e^{i\gamma_n}
    \end{equation}
    
    \begin{block}{Berry Phase (1984) \cite{Berry1984}}
        \begin{equation}
            \gamma_n = i\oint_{\mathcal{C}} \braket{n(R)}{\nabla_R n(R)} \cdot dR
        \end{equation}
        where $\mathcal{C}$ is the closed path in parameter space.
    \end{block}
    
    \vspace{0.3cm}
    
    \textbf{Properties:}
    \begin{itemize}
        \item Gauge invariant (physical observable)
        \item Geometric: depends only on path, not speed
        \item Can be written as flux of Berry curvature
    \end{itemize}
\end{frame}

\begin{frame}{Berry Connection and Curvature}
    \begin{block}{Berry Connection (Vector Potential)}
        \begin{equation}
            \vec{A}_n(R) = i\braket{n(R)}{\nabla_R n(R)}
        \end{equation}
    \end{block}
    
    \begin{block}{Berry Curvature (Field Strength)}
        \begin{equation}
            \vec{F}_n(R) = \nabla_R \times \vec{A}_n(R)
        \end{equation}
    \end{block}
    
    By Stokes' theorem:
    \begin{equation}
        \gamma_n = \oint_{\mathcal{C}} \vec{A}_n \cdot dR = \iint_{\mathcal{S}} \vec{F}_n \cdot d\vec{S}
    \end{equation}
    
    \vspace{0.3cm}
    
    \textbf{Analogy:} Berry phase is like the Aharonov-Bohm phase, but in parameter space rather than real space!
\end{frame}

% Section 7: Beyond the Simple Picture
\section{Advanced Topics}

\section{Advanced Topics}

\begin{frame}{Non-Adiabatic Transitions}
    When adiabaticity breaks down, transitions occur between energy levels.
    
    \begin{block}{Landau-Zener Formula}
        For a linear crossing: $E_1(t) = \alpha t$, $E_2(t) = -\alpha t + \Delta$
        
        Transition probability:
        \begin{equation}
            P_{1\to 2} = \exp\left(-\frac{\pi\Delta^2}{2\hbar\alpha}\right)
        \end{equation}
    \end{block}
    
    \vspace{0.3cm}
    
    \begin{itemize}
        \item Small gap $\Delta$ $\Rightarrow$ high transition probability
        \item Fast sweep (large $\alpha$) $\Rightarrow$ high transition probability
        \item Slow sweep (small $\alpha$) $\Rightarrow$ adiabatic regime, $P \to 0$
    \end{itemize}
    
    \vspace{0.3cm}
    
    \begin{alertblock}{Critical Insight}
        The Landau–Zener formula appears \textbf{everywhere}: in molecular physics,
        quantum computing, and condensed matter systems with avoided level crossings.
    \end{alertblock}
\end{frame}



\begin{frame}{Adiabatic Perturbation Theory}
    We can systematically expand in the adiabatic parameter $\epsilon = \tau_{\text{dyn}}/T$:
    
    \begin{equation}
        \ket{\psi(t)} = e^{i\theta_n(t)}\left[\ket{n(t)} + \epsilon\ket{\psi_n^{(1)}(t)} + \epsilon^2\ket{\psi_n^{(2)}(t)} + \cdots\right]
    \end{equation}
    
    First-order correction:
    \begin{equation}
        \ket{\psi_n^{(1)}(t)} = \sum_{m\neq n}\frac{\bra{m}\dot{\Hhat}\ket{n}}{(E_n - E_m)^2}\ket{m}
    \end{equation}
    
    \vspace{0.3cm}
    
    This shows:
    \begin{itemize}
        \item Transition amplitude $\propto 1/(E_n - E_m)^2$
        \item Small gaps make adiabatic evolution difficult
        \item Can estimate fidelity: $|\braket{n(t)}{\psi(t)}|^2 \approx 1 - \mathcal{O}(\epsilon^2)$
    \end{itemize}
\end{frame}

\begin{frame}{Generalization to Degenerate Levels}
    When eigenvalues are degenerate, the theorem must be modified:
    
    \begin{itemize}
        \item System can transition within degenerate subspace
        \item Need to consider the entire degenerate subspace
        \item Berry phase becomes a \textbf{matrix} (non-Abelian Berry connection)
    \end{itemize}
    
    \vspace{0.3cm}
    
    \begin{block}{Wilczek-Zee Connection \cite{Wilczek1984}}
        For degenerate subspace with basis $\{\ket{n_a(R)}\}$:
        \begin{equation}
            (A_i)_{ab} = i\braket{n_a}{\partial_i n_b}
        \end{equation}
        This is a matrix-valued gauge potential.
    \end{block}
    
    \vspace{0.3cm}
    
    Leads to non-Abelian geometric phases (important in topological quantum computation).
\end{frame}

%\begin{frame}{Discussion: Diabatic vs Adiabatic Evolution}
%    \begin{columns}
%    \column{0.5\textwidth}
%    \textbf{Diabatic Evolution:}
%    \begin{itemize}
%        \item Fast changes in $\Hhat(t)$
%        \item System doesn't track instantaneous eigenstates
%        \item Transitions between levels
%        \item Example: Sudden approximation
%    \end{itemize}
%    
%    \vspace{0.3cm}
%    
%    \textbf{When applicable:}
%    \begin{equation}
%        T \ll \frac{\hbar}{\Delta E}
%    \end{equation}
    
%    \column{0.5\textwidth}
%    \textbf{Adiabatic Evolution:}
%    \begin{itemize}
%        \item Slow changes in $\Hhat(t)$
%        \item System follows instantaneous eigenstate
%        \item No transitions (to first order)
%        \item Example: Born-Oppenheimer
%    \end{itemize}
%    
%    \vspace{0.3cm}
%    
%    \textbf{When applicable:}
%    \begin{equation}
%        T \gg \frac{\hbar}{\Delta E}
%    \end{equation}
%    \end{columns}
%    
%    \vspace{0.5cm}
%    
%    \begin{block}{Discussion Question}
%        In between these limits, how do we calculate the transition probability? 
        
%        \textbf{Answer:} Landau-Zener formula, perturbation theory, or numerical methods
%    \end{block}
%\end{frame}



% New Section
\begin{frame}{Application: Adiabatic Quantum Computation \cite{Farhi2000, Albash2018}}

    \begin{equation}
        \Hhat(t) = (1-s(t))\Hhat_{\text{initial}} + s(t)\Hhat_{\text{problem}}
    \end{equation}

    \begin{block}{Basic Idea}
        \begin{enumerate}
            \item Encode problem in Hamiltonian $\Hhat_{\text{problem}}$ whose ground state is the solution
            \item Start with simple Hamiltonian $\Hhat_{\text{initial}}$ with known ground state
            \item Slowly interpolate: $\Hhat(t) = (1-s(t))\Hhat_{\text{initial}} + s(t)\Hhat_{\text{problem}}$
            \item If evolution is adiabatic, system remains in ground state
            \item At $t = T$: measure to obtain solution
        \end{enumerate}
    \end{block}
    
    \vspace{0.3cm}
    
    \textbf{Runtime:} $T \sim \mathcal{O}(\hbar/\Delta_{\min}^2)$ where $\Delta_{\min}$ is minimum gap
    
    \vspace{0.3cm}
    
    \textbf{Challenge:} Gap can be exponentially small for some problems!
\end{frame}

\begin{frame}{Adiabatic Theorem in Quantum Annealing}
    \textbf{Success condition based on adiabatic theorem:}
    
    \begin{block}{Required Annealing Time}
        \begin{equation}
            T \gg \frac{\hbar}{\Delta_{\min}^2}
        \end{equation}
        where $\Delta_{\min}$ is the minimum energy gap during evolution
    \end{block}
    
    \vspace{0.3cm}
    
    \textbf{Problem-Dependent Gap:}
    \begin{itemize}
        \item Different problems $\Rightarrow$ different gap structures
        \item Gap often smallest near $t \approx T/2$ (mid-anneal)
        \item For many NP-hard problems: $\Delta_{\min} \sim e^{-\alpha N}$ (exponentially small!)
    \end{itemize}
    
    \vspace{0.3cm}
    
    \textbf{Practical Implications:}
    \begin{itemize}
        \item[$+$] Easy problems: Fast annealing possible (microseconds)
        \item[$-$] Hard problems: May need exponentially long annealing time
        \item[$\pm$] Thermal effects can help/hinder (quantum vs simulated annealing)
    \end{itemize}
\end{frame}

% Section: QUBO and Ising Models
\section{Quantum Annealing: QUBO and Ising Models}

\begin{frame}{Quantum Annealing Framework}
    \textbf{Goal:} Solve optimization problems using quantum adiabatic evolution
    
    \vspace{0.3cm}
    
    \begin{block}{Standard Approach}
        \begin{enumerate}
            \item Encode optimization problem in a Hamiltonian
            \item Prepare system in ground state of simple initial Hamiltonian
            \item Adiabatically evolve to problem Hamiltonian
            \item Measure final state to read out solution
        \end{enumerate}
    \end{block}
    
    \vspace{0.3cm}
    
    \textbf{Two Main Formulations:}
    \begin{itemize}
        \item \textbf{Ising Model:} Spin variables $s_i \in \{-1, +1\}$
        \item \textbf{QUBO:} Binary variables $x_i \in \{0, 1\}$
        \item These are equivalent and interconvertible
    \end{itemize}
    
    \vspace{0.3cm}
    
    \textbf{Examples:} Graph coloring, traveling salesman, portfolio optimization, protein folding \cite{Lucas2014}
\end{frame}

\begin{frame}{The Ising Model}
    \begin{block}{Ising Hamiltonian}
        \begin{equation}
            \hat{H}_{\text{Ising}} = \sum_{i<j} J_{ij} s_i s_j + \sum_i h_i s_i
        \end{equation}
        where $s_i \in \{-1, +1\}$ are spin variables (Pauli-z eigenvalues)
    \end{block}
    
    \vspace{0.3cm}
    
    \textbf{Parameters:}
    \begin{itemize}
        \item $J_{ij}$: Coupling between spins $i$ and $j$
        \begin{itemize}
            \item $J_{ij} > 0$: Ferromagnetic (spins want to align)
            \item $J_{ij} < 0$: Antiferromagnetic (spins want to anti-align)
        \end{itemize}
        \item $h_i$: Local field on spin $i$ (bias)
    \end{itemize}
    
    \vspace{0.3cm}
    
    %\textbf{Quantum Version:}
    %\begin{equation}
    %    \hat{H}_{\text{Ising}} = \sum_{i<j} J_{ij} \sigma_i^z \sigma_j^z + \sum_i h_i \sigma_i^z
    %\end{equation}
    
    This is the \textbf{problem Hamiltonian} in quantum annealing!
\end{frame}

\begin{frame}{Visualizing the Ising Model: Max-Cut Example}
    \begin{figure}
        \centering
        \includegraphics[width=0.75\textwidth]{figures/maxcut_example.png}
        \caption{\tiny Max-Cut example graph (problem instance)}
    \end{figure}
    
    \vspace{-0.3cm}
    
    \begin{itemize}
        \item \textbf{Left:} Original graph with 5 vertices and 6 edges
        \item \textbf{Right:} Optimal partition (red vs blue) achieves cut size = 5
        \item Only 1 edge (out of 6) remains within a partition
        \item Ising formulation: Antiferromagnetic couplings ($J_{ij} = 1$) favor opposite spins
    \end{itemize}
\end{frame}

%\begin{frame}{Energy Landscape: Solution Distribution}
%    \begin{center}
%        \includegraphics[width=0.45\textwidth]{figures/maxcut_distribution.png}
%    \end{center}
%    
%    \vspace{-0.3cm}
%    
%    \begin{itemize}
%        \item Out of $2^5 = 32$ possible partitions, only 2 achieve the optimal cut size
%        \item Most partitions achieve sub-optimal cuts (sizes 3-4)
%        \item \textbf{Challenge for quantum annealing:} Find the needle in the haystack!
%        \item Energy landscape roughness determines difficulty
%    \end{itemize}
%\end{frame}

\begin{frame}{QUBO: Quadratic Unconstrained Binary Optimization}
    \begin{block}{QUBO Formulation}
        Minimize:
        \begin{equation}
            f(x) = \sum_{i} Q_{ii} x_i + \sum_{i<j} Q_{ij} x_i x_j = x^T Q x
        \end{equation}
        where $x_i \in \{0, 1\}$ and $Q$ is the QUBO matrix
    \end{block}
    
    \vspace{0.3cm}
    
    \textbf{Why QUBO?}
    \begin{itemize}
        \item Many combinatorial optimization problems naturally expressed as QUBO
        \item NP-hard in general (finding global minimum)
        \item Can encode constraints via penalty terms
    \end{itemize}
    
    \vspace{0.3cm}
    
    \textbf{Examples:}
    \begin{itemize}
        \item Max-Cut: $\max \sum_{(i,j) \in E} (1 - x_i x_j)$
        \item Number Partitioning: $\min \left(\sum_i a_i x_i\right)^2$
        \item Graph Coloring: Use penalty for adjacent same-colored vertices
    \end{itemize}
\end{frame}

%\begin{frame}{Converting Between QUBO and Ising}
%    \textbf{The mapping is straightforward:}
%    
%    \begin{block}{QUBO $\rightarrow$ Ising}
%        Variable transformation: $x_i = \frac{1 - s_i}{2}$ where $s_i \in \{-1, +1\}$
%        
%        \vspace{0.2cm}
%        
%        Then: $x_i x_j = \frac{(1-s_i)(1-s_j)}{4} = \frac{1 - s_i - s_j + s_i s_j}{4}$
%    \end{block}
%    
%    \vspace{0.3cm}
%    
%    \begin{block}{Ising $\rightarrow$ QUBO}
%        Inverse transformation: $s_i = 1 - 2x_i$ where $x_i \in \{0, 1\}$
%        
%        \vspace{0.2cm}
%        
%        Then: $s_i s_j = (1-2x_i)(1-2x_j) = 1 - 2x_i - 2x_j + 4x_i x_j$
%    \end{block}
%    
%    \vspace{0.3cm}
%    
%    \textbf{Key Point:} The two formulations are equivalent - choose based on:
%    \begin{itemize}
%        \item Natural problem representation
%        \item Hardware constraints (quantum annealers often use Ising)
%        \item Software/solver availability
%    \end{itemize}
%\end{frame}

%\begin{frame}{Quantum Annealing Schedule}
%    \textbf{Standard Quantum Annealing Hamiltonian:}
%    
%    \begin{equation}
%        \hat{H}(t) = A(t) \hat{H}_{\text{driver}} + B(t) \hat{H}_{\text{problem}}
%    \end{equation}
%    
%    \vspace{0.3cm}
%    
%    where:
%    \begin{itemize}
%        \item $\hat{H}_{\text{driver}} = -\sum_i \sigma_i^x$ (transverse field - creates superposition)
%        \item $\hat{H}_{\text{problem}} = \sum_{i<j} J_{ij} \sigma_i^z \sigma_j^z + \sum_i h_i \sigma_i^z$ (Ising model)
%        \item $A(t)$, $B(t)$ are annealing schedules with $A(0) \gg B(0)$ and $A(T) \ll B(T)$
%    \end{itemize}
%    
%    \vspace{0.3cm}
%    
%    \textbf{Standard Linear Schedule:}
%    \begin{align}
%        A(t) &= A_0 \left(1 - \frac{t}{T}\right) \\
%        B(t) &= B_0 \frac{t}{T}
%    \end{align}
%    
%    \vspace{0.2cm}
%    
%    At $t=0$: Easy ground state $\ket{+}^{\otimes N}$ (all spins in $+x$ direction)
%    
%    At $t=T$: Ground state encodes solution to optimization problem
%\end{frame}

%\begin{frame}{Example: Max-Cut Problem}
%    \textbf{Problem:} Given graph $G = (V, E)$, partition vertices into two sets to maximize edges between sets.
%    
%    \vspace{0.3cm}
%    
%    \begin{columns}
%    \column{0.5\textwidth}
%    \textbf{QUBO Formulation:}
%    
%    Maximize:
%    \begin{equation}
%        f(x) = \sum_{(i,j) \in E} (x_i - x_j)^2
%    \end{equation}
%    
%    Expanding:
%    \begin{equation}
%        = \sum_{(i,j) \in E} (x_i + x_j - 2x_i x_j)
%    \end{equation}
%    
%    \vspace{0.2cm}
%    
%    For minimization (quantum annealer):
%    \begin{equation}
%       H_{\text{QUBO}} = -\sum_{(i,j) \in E} (x_i + x_j - 2x_i x_j)
%    \end{equation}
%    
%    \column{0.5\textwidth}
%    \textbf{Ising Formulation:}
%    
%    Using $s_i = 1 - 2x_i$:
%    
%    \begin{equation}
%        H_{\text{Ising}} = -\sum_{(i,j) \in E} (1 - s_i s_j)
%    \end{equation}
%    
%    Simplifies to:
%    \begin{equation}
%        H_{\text{Ising}} = \sum_{(i,j) \in E} s_i s_j + \text{const}
%    \end{equation}
%    
%    \vspace{0.1cm}
%
%    \textbf{Interpretation:}
%    \begin{itemize}
%        \item $s_i = +1$: vertex in set A
%        \item $s_i = -1$: vertex in set B
%        \item Minimize $\Rightarrow$ maximize cut
%   \end{itemize}
%    \end{columns}
%\end{frame}

%\begin{frame}{Embedding Problems on Quantum Annealers}
%    \textbf{Hardware Constraints:}
%    
%    Real quantum annealers (e.g., D-Wave) have limited connectivity:
%    
%    \vspace{0.3cm}
%    
%    \begin{columns}
%    \column{0.5\textwidth}
%    \textbf{Chimera Graph (D-Wave):}
%    \begin{itemize}
%        \item Unit cells of $K_{4,4}$ bipartite graphs
%        \item Not fully connected
%        \item Typical: 2000+ qubits
%        \item Connectivity: $\sim$6 per qubit
%    \end{itemize}
%    
%    \vspace{0.3cm}
%    
%    \textbf{Pegasus Graph (newer):}
%    \begin{itemize}
%        \item Higher connectivity ($\sim$15)
%        \item 5000+ qubits
%        \item Better for dense problems
%    \end{itemize}
%    
%    \column{0.5\textwidth}
%    \textbf{Minor Embedding:}
%    
%    Problem: Map logical problem graph to hardware graph
%    
%    \vspace{0.3cm}
%    
%    \textbf{Technique:} Chain multiple physical qubits to represent one logical qubit
%    
%    \vspace{0.3cm}
%    
%    \textbf{Challenges:}
%    \begin{itemize}
%        \item Uses more qubits
%        \item Requires strong ferromagnetic coupling for chains
%        \item Chain breaks $\Rightarrow$ errors
%    \end{itemize}
%    
%    \end{columns}
%\end{frame}

\begin{frame}[allowframebreaks]{References}
    \printbibliography
\end{frame}

\end{document}